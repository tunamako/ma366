\documentclass[letterpaper,12pt]{article}
\usepackage{setspace, amssymb}
\doublespacing
\begin{document}
\title{An Introduction to Manifolds}
\author{Michael Whalen}
\date{December 8, 2017}
\maketitle


Intuitively, manifolds are the topological structures that we have been investigating and working with all semester. However we've used different names for each specific manifold we've used, such as 'sphere', 'disk', or 'torus'. With all we've experienced, the definition of a manifold is relatively simple: a \textbf{Manifold} is a topological space locally modeled in $\mathbb{R}^{n}$.\cite[p.109]{Thurston}Essentially, at any given point on a manifold you can find a local area around it that is homeomorphic to $\mathbb{R}^{n}$. 

To add specificity, all manifolds are necessarily Hausdorff and have a countable topological basis. One-dimensional manifolds are called \textbf{Curves}, examples of which are functions like $f(x) = x^{2}$ and $S^{1}$. Two-dimensional manifolds are called \textbf{Surfaces} which we've seen as $S^{2}$, the disk, and the hollow torus. 

Like Connectedness and Compactness of topological spaces we've studied, manifolds have the property of \textbf{Orientability}, which defines whether it's possible to 'walk' along every loop of the manifold and arrive to where you began without being on the other 'side' of the manifold. Consider the Mobius strip, where if you start at a point and walk along a loop in strip, you will eventually arrive to the same point you started, but upside down. When this happens, the manifold you're walking on is \textbf{non-orientable}. Furthermore, orientability is preserved by homeomorphism. This can be used to verify if a 2-manifold is orientable by seeing if there exists an embedding of the Mobius band inside it. 

Another property of manifolds is \textbf{Smoothness}, which describes the differentiability of areas of the manifold. In simplest terms, a manifold is smooth if it is locally differentiable up to any order of differentiation. \textbf{Smooth Maps} between manifolds have all the same properties of continuous functions we've studied, with the extra caveat that they preserve differentiability. \cite[p.61]{Tu} This concept provides an incredible amount of depth for study with wide applications in the fields of physics and cosmology.

We can also perform operations on manfolds. So far in class we've seen mappings, homeomorphisms, product spaces, and quotient spaces, which all can be applied to general manifolds. There are also \textbf{Connected Sums}, a kind of 'gluing together' of manifolds along a cut boundary. In the case of tori, multiple can be glued together to create extended 'figure-eight' surfaces, each associated with a \textbf{Genus}, which counts how many ways you can cut the surface without rendering it disconnected.

With lower-dimension manifolds like the ones we have worked on it's easy to visualize and keep a mental image of them as we perform operations and mappings on them. However, moving into the realm of 3-manifolds that becomes difficult. For example, the 3-torus: sitting inside one would be like sitting in a room where the walls are mirrors, letting you see infinitely in any direction. But instead of the walls reflecting an image, they simply display the image as seen from the other side of the room, so that looking to the right, you would see the left side of your body. \cite[p.31]{Thurston}. Similar to how we generated a 2-torus by gluing the sides of a square together, we can generate a 3-torus by gluing the sides of a cube together.

The study of manifolds can also be connected to group theory in many ways, one of which is by investigating \textbf{Lie Groups}. A lie group G is a manifold which is also a group that satisfies two operations, multiplication and inversion, such that smoothness is preserved. \cite[p.164]{Tu} Interestingly, homomorphisms between lie groups are just smooth maps between their manifolds. \textbf{Lie Subgroups} are naturally  corresponding to the concept of submanifolds, where a submanifold of a lie group is a lie subgroup if smoothness across operations is preserved. 

\bibliography{references}
\bibliographystyle{ieeetr}
\end{document}