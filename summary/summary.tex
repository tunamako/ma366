\documentclass[letterpaper,12pt]{article}
\newcommand{\forceindent}{\leavevmode{\parindent=1em\indent}}

\usepackage{indentfirst,amssymb}
\begin{document}
\title{MA 366 Course Summary}
\author{Michael Whalen}
\date{December 04 2017}
\maketitle

\setlength{\parindent}{0pt}


\section*{Definitions}

	\textbf{n-manifold:}
		An object that is locally similar to n-dimensional Euclidean space.
	\newline{}

	\textbf{Arbitrary Union:}
		The union of a collection of sets
	\newline{}

	\textbf{Equivalence Relation:}
		A relation of a set onto itself satisfying reflexivity, symmetry, and transitivity
	\newline{}

	\textbf{Surjective Function:}
		A function from one set to another such that each element in the range is 'hit' by the function; also known as 'onto'
	\newline{}

	\textbf{Injective Function:}
		A function from one set to another such that each element in the domain has a corresponding element in the range; also known as 'one to one'
	\newline{}

	\textbf{Topology:}
		Let X be a set. A topology $\tau$ on X is a collection of subsets of X such that:
		\begin{enumerate}
			\item X and the empty set are in $\tau$
			\item Arbitrary unions of open sets are open
			\item Finite intersections of open sets are open
		\end{enumerate}

	\textbf{Discrete Topology:}
		The topology on a set X such that every subset of X is open in the topology
	\newline{}

	\textbf{Finite Compliment Topology:}
		The topology where open sets are defined to be those sets who have a compliment of finite size
	\newline{}

	\textbf{Lower Limit Topology:}
		The topology generated by a basis whose elements are of the form [a, b) $\forall$ a,b $\in$ $\mathbb{R}$ such that a $<$ b. 
	\newline{}

	\textbf{Basis:}
		A basis B is a collection of open subsets of a set X such that:
		\begin{enumerate}
			\item B covers X

			\item If an element x is in the intersection of two basis elements, then there is a smaller basis element that contains x and is contained in the intersection
		\end{enumerate}

	\textbf{Topology Generation:}
		A basis will 'generate' a topology by defining open sets to be the union of basis elements.
	\newline{}

	\textbf{Hausdorff Space:}
		A topological space (X, T) is Hausdorff if for every x and y in X, there are some u$_{x}$ and u$_{y}$ neighborhoods about x and y which are disjoint from one another.
	\newline{}

	\textbf{Interior:}
		For a set A, the interior of A denoted by $\mathring{A}$ is the union of open sets contained in A. Intuitively, it is the largest open set contained in A.
	\newline{}

	\textbf{Closure:}
		For a set A, the closure of A denoted by $\overline{A}$ is the intersection of all closed sets containing A. Intuitively, it is the smallest closed set containing A.
	\newline{}

	\textbf{Dense:}
		A set A is dense in X if $\overline{A}$ = X.
	\newline{}

	\textbf{Limit Point:}
		For a set A $\subset$ X, a point x $\in$ X is a limit point of A if every open neighborhood about x intersects A. The set of limit points of A is denoted by $A^{\prime}$.
	\newline{}

	\textbf{Boundary:}
		The boundary of a set A is defined as $\delta A = \overline{A}\setminus\mathring{A}$.
	\newline{}

	\textbf{Convergence:}
		A sequence {X$_{n}$} converges to \emph{x} if there exists some $N>0$ such that for every open neighborhood \emph{u} of \emph{x}, $\forall\emph{n} > \emph{N}$, $X_{n}$ is inside \emph{u}.
	\newline{}

	\textbf{Subspace Topology:}
		Let Y $\subset$ X. The topology $\tau_{y}$ defines open sets in Y to be all sets that intersect an open set in X.
	\newline{}

	\textbf{Product Space:}
		Let X and Y be topological spaces. The product space $X\times Y$ has a topology whose basis is {$u\times v$ where \emph{u} is open in X and \emph{v} is open in Y}.
	\newline{}

	\textbf{Quotient Spaces:}
		Let X be a topological space and A $\subset$ X be a set. Let $\phi:X\mapsto A$ be a surjective map. Choose a subset \emph{U}$\subset$A to be open in A if and only if $\phi^{-1}(U)$ is open in X.

		The resultant collection of open sets in A is the quotient topology induced by $\phi$. The function $\phi$ is called a quotient map. The topological space A is called a quotient space. The quotient map effectively "glues" subsets of X together.
	\newline{}

	\textbf{Continuous Functions:}
		Let $X$ and $Y$ be topological spaces. A function $f:X\mapsto Y$ is continuous if $f^{-1}(V)$ is open for every open set $V\subset Y$.
	\newline{}

	\textbf{Homeomorphisms:}
		Let $X$ and $Y$ be topological spaces, and let $f:X\mapsto Y$ be a bijection. If both $f$ and $f^{-1}$ are continuous functions, then $f$ is said to be a homeomorphism. If a homeomorphism exists between two spaces, they are considered to be homeomorphic, and more colloquially, 'topologically equivalent'.
	\newline{}

	\textbf{Embeddings:}
		An embedding of $X$ in $Y$ is a function $f:X\mapsto Y$ that maps $X$ homeomorphically to the subspace $f(X)$ in $Y$.
	\newline{}

	\textbf{Metrics:}
		A metric on a set $X$ is a function $d:X\times X\mapsto\mathbb{R}$ where the following is true:
		\begin{enumerate}
			\item $d(x,y)\geq0$ $\forall x,y\in X$; equality holds if and only if $x=y$.
			\item $d(x,y)=d(y,x)$ $\forall x,y\in X$.
			\item $d(x,y)+d(y,z)\geq d(x,z)$ $\forall x,y,z\in X$.
		\end{enumerate}

	\textbf{Connectedness:}
		A topological space $X$ is considered to be connected if there does not exist a pair of disjoint nonempty open sets whose union is $X$. If $X$ is not connected, it is considered to be disconnected, and the pair of disjoint sets that make it up are called a separation of $X$.
	\newline{}

	\textbf{Cutset:}
		Let $X$ be a topological space. A cutset is a set $S\subset X$ such that $X\setminus S$ is disconnected.
	\newline{} 

	\textbf{Cover:}
		Let $A$ be a subset of a topological space $X$. Let $O$ be a collection of subsets of $X$. If $A\sqcup_\alpha O_\alpha$, then $O$ covers A. If all $O_\alpha$'s are open, then it is considered to be an open cover.
	\newline{}

	\textbf{Compactness:}
		A set $A$ is compact if and only if every open cover of $A$ contains a finite sub-cover.
	\newline{}


\section*{Theorems}

	\textbf{Brower's Fixed-Point Theorem:}
		Every continuous map of a closed disk to itself has a fixed point.
	\newline{}

	\textbf{Union Lemma:}
		Let X be a set. For every x$\in$X, $\exists u_x$ where $u_x$ is open and contains \emph{x}. Then X is equal to the arbitrary union of these neighborhoods.
	\newline{}

	\textbf{1.19:}
		If a topological space X is Hausdorff, every single-point subset $\{p\}\subset$ X is closed.
	\newline{}

	\textbf{2.2:}
	\begin{enumerate}
			\item $A$ is open if and only if $A$ = $\mathring{A}$
			\item $A$ is closed if and only if $A$ = $\overline{A}$
	\end{enumerate}

	\textbf{2.5:}
		A point \emph{x} is contained in $\overline{A}$ if and only if every neighborhood of \emph{x} intersects $A$.
	\newline{}

	\textbf{2.9:}
		A set $A$ is closed if and only if $A^{\prime}\subset A$.
	\newline{}

	\textbf{2.11:}
		Let $A\subset \mathbb{R}^{n}$. If $x\subset A^{\prime}$, then there is a sequence of points in $A$ that converges to $x$.
	\newline{}

	\textbf{2.12:}
		If $X$ is a Hausdorff space, then every convergent sequence of points in $X$ converges to a unique point in $X$.
	\newline{}

	\textbf{2.14:}
		For a set $A\subset X$, let $x\in X$. $x$ is in the boundary of $A$ if and only if every neihborhood of $x$ intersects both $A$ and $X\setminus A$.
	\newline{}

	\textbf{2.15:}
		Let $A$ be a subset of a topological space $X$. The following are then true:
	\begin{enumerate}
		\item $\delta A$ is closed.
		\item $\delta A=\overline{A}\cap\overline{X\setminus A}$.
		\item $\delta A\cap\mathring{A}=\varnothing$.
		\item $\delta A\cup\mathring{A}=\overline{A}$.
		\item $\delta A\subset A$ if and only if $A$ is closed.
		\item $\delta A\cap=\varnothing$ if and only if $A$ is open.
		\item $\delta A=\varnothing$ if and only if $A$ is clopen.
	\end{enumerate}

	\textbf{3.4:}
		Let $X$ be a topological space, and let $Y\subset X$ have the subspace topology. Then $C\subset Y$ is closed in $Y$ if and only if $C = D\cap Y$ for some closed set $D$ in $X$.
	\newline{}

	\textbf{3.8:}
		If $C$ is a basis for $X$ and $D$ is a basis for $Y$, then $\mathcal{E}=\{c\times d : c\in C$ and $d\in D\}$ is a basis that generates the product topology on $X\times Y$.
	\newline{}

	\textbf{3.10:}
		Let $A$ and $B$ be subsets of topological spaces $X$ and $Y$, respectively. Then $\mathring{A\times B}=\mathring{A}\times\mathring{B}$.
	\newline{}

	\textbf{4.3:}
		Let $X$ and $Y$ be topological spaces and $\beta$ be a basis for the topology on $Y$. Then $f:X\mapsto Y$ is continuous if and only if $f^{-1}(B)$ is open in $X$ for every $B\subset\beta$.
	\newline{}

	\textbf{4.7:}
		Assume that $f:X\mapsto Y$ is continuous. If a sequence $(x_1, x_2,\ldots)$in $X$ converges to a point $x$, then the sequence $(f(x_1),f(x_2)\ldots)$ in $Y$ converges to $f(x)$.
	\newline{}

	\textbf{4.9:}
		Let $f:X\mapsto Y$ and $g:Y\mapsto Z$ be continuous. Then the composition function $g\circ f:X\mapsto Z$ is continuous.
	\newline{}

	\textbf{4.17:}
		If $f:X\mapsto Y$ is a homeomorphism and $X$ is Hausdorff, then $Y$ is Hausdorff.
	\newline{}

	\textbf{Properties of Connectedness:}
	\begin{enumerate}
		\item Continuous functions preserve connectedness.
		\item Homeomorphisms preserve connectedness.
		\item Products of connected spaces are connected.
		\item Adding limit points to a connected set does not affect connectedness.
		\item The union of connected sets who all share a common intersection is connected.
	\end{enumerate}

	\textbf{Jordan Curve Theorem:}
		Any simple, closed curve in $\mathbb{R}^{2}$ splits $\mathbb{R}^{2}$ into two pieces.
	\newline{}

	\textbf{Properties of Compactness}
	\begin{enumerate}
		\item Continuous functions preserve compactness.
		\item Homeomorphisms preserve compactness.
		\item Product and Quotient spaces preserve compactness.
	\end{enumerate}

	\textbf{Extreme Value Theorem:}
		Let $X$ be compact and $f:X\mapsto\mathbb{R}$ be continuous. Then $f$ takes on a maximum value and a minimum value on $X$.

\end{document}